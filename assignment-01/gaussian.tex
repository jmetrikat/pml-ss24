\documentclass[12pt]{scrartcl}

\usepackage{geometry}
\geometry{top=15mm, headheight=70pt, headsep=25pt, inner=1.25in, outer=1.25in}

\usepackage[utf8]{inputenc}
\usepackage[ngerman]{babel}
\usepackage{
	enumerate,
	amsthm,
	xcolor,
	graphicx,
	amsmath,
	amssymb,
	latexsym,
	framed,
	algorithmicx,
	algorithm,
	tikz,
	stmaryrd,
	enumitem
}
\usetikzlibrary{positioning, calc, arrows.meta, decorations.pathreplacing}

\usepackage{fancyhdr}
\usepackage{graphicx}
\pagestyle{fancy}
\lhead{\makeNames}
\rhead{Abgabe Aufgabenblatt \makeSheetNumber}
\cfoot{\thepage}

\newcommand{\makeheader}{
\begin{framed}
	\noindent
	\makeNames \hfill Probabilistic Machine Learning
	\begin{center}
		\Large{\textbf{Assignment \makeSheetNumber}}
	\end{center}
	\hfill Sommersemester 2024
\end{framed}
}

\newcommand{\names}[1]{\newcommand{\makeNames}{#1}}
\newcommand{\group}[1]{\newcommand{\makeGroup}{#1}}
\newcommand{\sheetNumber}[1]{\newcommand{\makeSheetNumber}{#1}}
%%% End Header Definitions %%%

\theoremstyle{definition}
\newtheorem{definition}{Definition}
\newtheorem{lemma}{Lemma}
\newtheorem{satz}{Satz}

\newtheoremstyle{exercise}% name of the style to be used
{2em}% measure of space to leave above the theorem.
{1em}% measure of space to leave below the theorem.
{\addtolength{\leftskip}{\parindent}}% name of font to use in the body of the theorem
{-\parindent}% measure of space to indent
{\bfseries}% name of head font
{:}% punctuation between head and body
{\newline}% space after theorem head; " " = normal interword space
{}% Manually specify head
\theoremstyle{exercise}
\newtheorem{exercise}{Aufgabe}
\newtheorem*{exercise*}{Aufgabe}
\newtheorem*{solution}{Lösung}
\newtheorem*{points}{Bepunktung}
\newtheorem{bonus}{Bonusaufgabe}
\renewcommand{\thebonus}{B\arabic{bonus}}

\renewcommand{\labelenumi}{(\alph{enumi})}
\renewcommand{\labelenumii}{(\roman{enumii})}

\newcommand*{\bigO}{\mathrm{O}}
\newcommand*{\Oh}{\mathrm{O}}
\newcommand*{\bigOmega}{\Omega}
\newcommand*{\smallO}{\mathrm{o}}
\newcommand*{\smallOmega}{\upomega}

\newcommand*{\natnum}{\mathbb{N}}
\newcommand*{\integers}{\mathbb{Z}}
\newcommand*{\rationals}{\mathbb{Q}}
\newcommand*{\realnum}{\mathbb{R}}

\newcommand{\comment}[1]{\textcolor{blue}{[#1]}}

\newcommand{\set}[2]{\{ #1 \mid #2 \}}

\newcommand{\CalA}{\mathcal{A}}
\newcommand{\CalC}{\mathcal{C}}
\newcommand{\CalP}{\mathcal{P}}
\newcommand{\CalM}{\mathcal{M}}
\newcommand{\CalR}{\mathcal{R}}
\newcommand{\CalS}{\mathcal{S}}

\newcommand{\eqnComment}[2]{\underset{{\scriptstyle \text{#1}}}{#2}}

%%%%%%%%%%%%%%%%%%%%%%%%%%%%%%%%%%%% Your Data goes here %%%%%%%%%%%%%%%%%%%%%%%%%%%%%%%%%%%
\names{Jannis Metrikat, Philipp Hendel} % Hier die Name aller Abgebenden
\sheetNumber{1} % Hier die Zahl des Aufgabenblatts

%%%%%%%%%%%%%%%%%%%%%%%%%%%%%%%%%%%% Your Definitions go here %%%%%%%%%%%%%%%%%%%%%%%%%%%%%%

%%%%%%%%%%%%%%%%%%%%%%%%%%%%%%%%%%%%%%%%%%%%%%%%%%%%%%%%%%%%%%%%%%%%%%%%%%%%%%%%%%%%%%%%%%%%
\begin{document}
	\thispagestyle{empty}
	\makeheader

	% ---------------------------------------------------------------------------

	\begin{exercise*}
		\

		\begin{enumerate}
			\item Show that $\mathcal{N}(x;\mu,\sigma^{2})= G(x; \frac{\mu}{\sigma^{2}}
				, \frac{1}{\sigma^{2}})$:

				\begin{align*}
					G(x, \frac{\mu}{\sigma^{2}}, \frac{1}{\sigma^{2}}) & = \sqrt{\frac{\frac{1}{\sigma^{2}}}{2 \pi}}\cdot \exp\Big(- \frac{(\frac{\mu}{\sigma^2})^{2}}{2 \frac{1}{\sigma^{2}}}\Big) \cdot \exp\Big(\frac{\mu}{\sigma^{2}}\cdot x - \frac{1}{\sigma^{2}}\frac{x^{2}}{2}\Big) \\
					                                                   & = \sqrt{\frac{1}{2 \pi \sigma^{2} }}\cdot \exp\Big(- \frac{\mu^{2}}{2\sigma^{2}}+ \frac{2\mu x}{2\sigma^{2}}- \frac{x^{2}}{2 \sigma^{2}}\Big)                                                                      \\
					                                                   & = \sqrt{\frac{1}{2 \pi \sigma^{2} }}\cdot \exp\Big(- \frac{1}{2} \cdot \frac{(x - \mu)^{2}}{\sigma^{2}}\Big)                                                                                                       \\
					                                                   & = \mathcal{N}(x;\mu,\sigma^{2})
				\end{align*}

			\item Show that $G(x; \tau, \rho) = \mathcal{N}(x; \frac{\tau}{\rho},\frac{1}{\rho}
				)$:

				\begin{align*}
					\mathcal{N}(x; \frac{\tau}{\rho},\frac{1}{\rho}) & = \sqrt{\frac{1}{2 \pi \frac{1}{\rho} }}\cdot \exp\Big(- \frac{1}{2} \cdot \frac{(x - \frac{\tau}{\rho})^{2}}{\frac{1}{\rho}}\Big)             \\
					                                                 & = \sqrt{\frac{\rho}{2 \pi}}\cdot \exp\Big(- \frac{1}{2} \cdot \frac{x^{2} - \frac{2 x \tau}{\rho}+ \frac{\tau^2}{\rho^2}}{\frac{1}{\rho}}\Big) \\
					                                                 & = \sqrt{\frac{\rho}{2 \pi}}\cdot \exp\Big(-\frac{x^{2} \rho}{2}+ x \tau - \frac{\tau^{2}}{2\rho}\Big)                                          \\
					                                                 & = \sqrt{\frac{\rho}{2 \pi}}\cdot \exp\Big(-\frac{\tau^{2}}{2\rho}\Big) \cdot \exp\Big(\tau \cdot x -\rho \cdot \frac{x^{2} }{2}\Big)           \\
					                                                 & = G(x; \tau, \rho)
				\end{align*}

			\item Write down the formulas for the densities
				\begin{enumerate}
					\item
						\begin{equation*}
							G(x; \tau_{1}, \rho_{1}) = \sqrt{\frac{\rho_{1}}{2 \pi}}\cdot \exp\Big
							(-\frac{\tau_{1}^{2}}{2\rho_{1}}\Big) \cdot \exp\Big(\tau_{1} \cdot
							x - \rho_{1} \cdot \frac{x^{2}}{2}\Big)
						\end{equation*}

					\item
						\begin{equation*}
							G(x; \tau_{2}, \rho_{2}) = \sqrt{\frac{\rho_{2}}{2 \pi}}\cdot \exp\Big
							(-\frac{\tau_{2}^{2}}{2\rho_{2}}\Big) \cdot \exp\Big(\tau_{2} \cdot
							x - \rho_{2} \cdot \frac{x^{2}}{2}\Big)
						\end{equation*}

					\item
						\begin{equation*}
							G(x; \tau_{1} + \tau_{2}, \rho_{1} + \rho_{2}) = \sqrt{\frac{\rho_{1}
							+ \rho_{2}}{2 \pi}}\cdot \exp\Big(-\frac{(\tau_{1} + \tau_{2})^{2}}{2(\rho_{1}
							+ \rho_{2})}\Big) \cdot \exp\Big((\tau_{1} + \tau_{2}) \cdot x - \rho
							_{1} + \rho_{2} \cdot \frac{x^{2}}{2}\Big)
						\end{equation*}

					\item
						\begin{equation*}
							\mathcal{N}(\frac{\tau_{1}}{\rho_{1}}; \frac{\tau_{2}}{\rho_{2}}, \frac{1}{\rho_{1}}
							+ \frac{1}{\rho_{2}}) = \frac{1}{\sqrt{2 \pi (\frac{1}{\rho_{1}} +
							\frac{1}{\rho_{2}})}}\cdot \exp\Big(- \frac{1}{2} \cdot \frac{(\frac{\tau_1}{\rho_1}-
							\frac{\tau_2}{\rho_2})^{2}}{\frac{1}{\rho_{1}} +
							\frac{1}{\rho_{2}}}\Big)
						\end{equation*}
				\end{enumerate}

			\item \scriptsize
				\begin{align*}
					G(x; \tau_{1}, \rho_{1}) \cdot G(x; \tau_{2}, \rho_{2})                                                                                                                                                                                                                                                                                                                                                                                                            &  \\
					= G(x; \tau_{1} + \tau_{2}, \rho_{1} + \rho_{2}) \cdot N(\frac{\tau_{1}}{\rho_{1}}; \frac{\tau_{2}}{\rho_{2}}, \frac{1}{\rho_{1}}+ \frac{1}{\rho_{2}})                                                                                                                                                                                                                                                                                                             &  \\
					\bigg(\sqrt{\frac{\rho_{1}}{2 \pi}}\cdot \exp\big(-\frac{\tau_{1}^{2}}{2\rho_{1}}\big) \cdot \exp\big(\tau_{1} \cdot x - \rho_{1} \cdot \frac{x^{2}}{2}\big)\bigg) \cdot \bigg(\sqrt{\frac{\rho_{2}}{2 \pi}}\cdot \exp\big(-\frac{\tau_{2}^{2}}{2\rho_{2}}\big) \cdot \exp\big(\tau_{2} \cdot x - \rho_{2} \cdot \frac{x^{2}}{2}\big)\bigg)                                                                                                                        &  \\
					= \bigg( \sqrt{\frac{\rho_{1} + \rho_{2}}{2 \pi}}\cdot \exp\big(-\frac{(\tau_{1} + \tau_{2})^{2}}{2(\rho_{1} + \rho_{2})}\big) \cdot \exp\big((\tau_{1} + \tau_{2}) \cdot x - \rho_{1} + \rho_{2} \cdot \frac{x^{2}}{2}\big) \bigg) \cdot \bigg( \frac{1}{\sqrt{2 \pi (\frac{1}{\rho_{1}} + \frac{1}{\rho_{2}})}}\cdot \exp\big(- \frac{1}{2} \cdot \frac{(\frac{\tau_1}{\rho_1}- \frac{\tau_2}{\rho_2})^{2}}{\frac{1}{\rho_{1}} + \frac{1}{\rho_{2}}}\big) \bigg) &  \\
					\sqrt{\frac{\rho_{1} \rho2}{4 \pi^{2}}}\cdot \exp\bigg(-\frac{\tau_{1}^{2}}{2\rho_{1}}-\frac{\tau_{2}^{2}}{2\rho_{2}}+ (\tau_{1} + \tau_{2}) \cdot x - \frac{(\rho_{1} + \rho_{2})x^{2}}{2}\bigg)                                                                                                                                                                                                                                                                  &  \\
					= \sqrt{\frac{\frac{\rho_{1} + \rho_{2}}{2 \pi}}{\frac{2 \pi \rho1_{1} + 2 \pi \rho_{2}}{\rho_{1} \rho_{2}}}}\cdot \exp\bigg(-\frac{(\tau_{1} + \tau_{2})^{2}}{2(\rho_{1} + \rho_{2})}+ (\tau_{1} + \tau_{2}) \cdot x - \frac{(\rho_{1} + \rho_{2})x^{2}}{2}- \frac{1}{2} \cdot \frac{(\frac{\tau_1}{\rho_1}- \frac{\tau_2}{\rho_2})^{2}}{\frac{1}{\rho_{1}} + \frac{1}{\rho_{2}}}\bigg)                                                                           &  \\
					\sqrt{\frac{\rho_{1} \rho2}{4 \pi^{2}}}\cdot \exp\bigg((\tau_{1} + \tau_{2}) \cdot x - \frac{(\rho_{1} + \rho_{2})x^{2}}{2}-\frac{\tau_{1}^{2}}{2\rho_{1}}-\frac{\tau_{2}^{2}}{2\rho_{2}}+ \frac{(\tau_{1} + \tau_{2})^{2}}{2(\rho_{1} + \rho_{2})}\bigg)                                                                                                                                                                                                          &  \\
					= \sqrt{\frac{(\rho_{1} + \rho_{2}) \cdot \rho_{1} \rho_{2}}{4 \pi^{2} (\rho_{1} + \rho_{2})}}\cdot \exp\bigg((\tau_{1} + \tau_{2}) \cdot x - \frac{(\rho_{1} + \rho_{2})x^{2}}{2}- \frac{1}{2} \cdot \frac{(\frac{\tau_1}{\rho_1}- \frac{\tau_2}{\rho_2})^{2}}{\frac{1}{\rho_{1}} + \frac{1}{\rho_{2}}}\bigg)                                                                                                                                                     &  \\
					\sqrt{\frac{\rho_{1} \rho2}{4 \pi^{2}}}\cdot \exp\bigg((\tau_{1} + \tau_{2}) \cdot x - \frac{(\rho_{1} + \rho_{2})x^{2}}{2}- \frac{1}{2}(\frac{\tau_{1}^{2}\rho_{2} + \tau_{2}^{2} \rho_{1}}{\rho_{1} \rho_{2}}- \frac{(\tau_{1} + \tau_{2})^{2}}{(\rho_{1} + \rho_{2})})\bigg)                                                                                                                                                                                    &  \\
					= \sqrt{\frac{\rho_{1} \rho_{2}}{4 \pi^{2}}}\cdot \exp\bigg((\tau_{1} + \tau_{2}) \cdot x - \frac{(\rho_{1} + \rho_{2})x^{2}}{2}- \frac{1}{2} \cdot \frac{(\frac{\tau_1}{\rho_1}- \frac{\tau_2}{\rho_2})^{2}}{\frac{\rho_{1} + \rho_{2}}{\rho_{1} \rho_{2}}}\bigg)                                                                                                                                                                                                 &  \\
					\sqrt{\frac{\rho_{1} \rho2}{4 \pi^{2}}}\cdot \exp\bigg((\tau_{1} + \tau_{2}) \cdot x - \frac{(\rho_{1} + \rho_{2})x^{2}}{2}- \frac{1}{2}((\frac{\rho_{1} + \rho_{2}) (\tau_{1}^{2}/\rho_{1} + \tau_{2}^{2} / \rho_{2})- (\tau_{1} + \tau_{2})^{2}}{\rho_{1} + \rho_{2}}\bigg)                                                                                                                                                                                      &  \\
					= \sqrt{\frac{\rho_{1} \rho_{2}}{4 \pi^{2}}}\cdot \exp\bigg((\tau_{1} + \tau_{2}) \cdot x - \frac{(\rho_{1} + \rho_{2})x^{2}}{2}- \frac{1}{2} \cdot \frac{(\frac{\tau_1}{\rho_1}- \frac{\tau_2}{\rho_2})^{2} \rho_{1} \rho_{2}}{\rho_{1} + \rho_{2}}\bigg)                                                                                                                                                                                                         &  \\
					\sqrt{\frac{\rho_{1} \rho2}{4 \pi^{2}}}\cdot \exp\bigg((\tau_{1} + \tau_{2}) \cdot x - \frac{(\rho_{1} + \rho_{2})x^{2}}{2}- \frac{1}{2}(\frac{\tau_{1}^{2} + \frac{\rho_1 \tau_2^2}{\rho_2}+ \frac{\rho_2 \tau_1^2}{\rho_1}+ \tau_{2}^{2} - \tau_{1}^{2} - 2\tau_{1}\tau_{2} - \tau_{2}^{2}}{\rho_{1} + \rho_{2}}\bigg)                                                                                                                                           &  \\
					= \sqrt{\frac{\rho_{1} \rho_{2}}{4 \pi^{2}}}\cdot \exp\bigg((\tau_{1} + \tau_{2}) \cdot x - \frac{(\rho_{1} + \rho_{2})x^{2}}{2}- \frac{1}{2} \cdot \frac{\frac{\tau_1^2\rho_2}{\rho_1}- 2 \tau_{1} \tau_{2} + \frac{\tau_2^2\rho_1}{\rho_2}}{\rho_{1} + \rho_{2}}\bigg)                                                                                                                                                                                           &  \\
					\sqrt{\frac{\rho_{1} \rho2}{4 \pi^{2}}}\cdot \exp\bigg((\tau_{1} + \tau_{2}) \cdot x - \frac{(\rho_{1} + \rho_{2})x^{2}}{2}- \frac{1}{2}(\frac{\frac{\tau_1^2\rho_2}{\rho_1}- 2 \tau_{1} \tau_{2} + \frac{\tau_2^2\rho_1}{\rho_2}}{\rho_{1} + \rho_{2}}\bigg)                                                                                                                                                                                                      &  \\
					= \sqrt{\frac{\rho_{1} \rho_{2}}{4 \pi^{2}}}\cdot \exp\bigg((\tau_{1} + \tau_{2}) \cdot x - \frac{(\rho_{1} + \rho_{2})x^{2}}{2}- \frac{1}{2} \cdot \frac{\frac{\tau_1^2\rho_2}{\rho_1}- 2 \tau_{1} \tau_{2} + \frac{\tau_2^2\rho_1}{\rho_2}}{\rho_{1} + \rho_{2}}\bigg)
				\end{align*}
		\end{enumerate}
	\end{exercise*}
\end{document}
